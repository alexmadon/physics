\documentstyle[french]{article}
\pagestyle{empty}
\textheight 250mm
\textwidth      142mm
\evensidemargin -6mm
\oddsidemargin  -6mm
%\topmargin      -0.5truecm

%%%%%
%\renewcommand{\baselinestretch}{2}   


   
\begin{document}

\begin{center}
{\large \bf R\'esum\'e}
\end{center}

Le probl\`eme abord\'e dans ce travail est de fournir des mod\`eles
pour la route vers la turbulence observ\'ee dans un syst\`eme
exp\'erimental. Le syst\`eme consid\'er\'e est un plasma de d\'echarge
confin\'e magn\'etiquement. Un diagnostic spatio--temporel est
assur\'e par une couronne de 64 sondes de Langmuir. Lorsqu'un des
param\`etres de contr\^ole de l'exp\'erience (une tension) augmente, on
observe diff\'erents r\'egimes, du  plasma stable \`a l'\'etat
de turbulence faible spatio-temporelle. Une m\'ethode d'analyse des
donn\'ees est propos\'ee. Elle s'appuie sur la d\'ecomposition
bi--orthogonale et consiste \`a identifier dans les
signaux des ondes modul\'ees. La description du comportement du
syst\`eme en termes 
d'ondes modul\'ees permet de d\'etecter des bifurcations qui n'avaient
pas \'et\'es observ\'ees par les m\'ethodes classiques. Un mod\`ele
pour le ph\'enom\`ene de doublement de vitesse observ\'e dans les
signaux exp\'erimentaux est propos\'e. Il est bas\'e sur des
modulations purement spatiales.
Un deuxi\`eme mod\`ele est propos\'e afin de
d\'ecrire la route compl\`ete vers la turbulence. Plus
pr\'ecis\'ement, nous donnons un syst\`eme dynamique qui d\'ecrit
l'\'evolution temporelle des amplitudes des ondes modul\'ees et qui
peut \^etre consid\'er\'e comme une forme normale pour la s\'equence
de bifurcations observ\'ees: 1) apparition d'une onde modul\'ee 2) un
\'etat quasi--p\'eriodique \`a quatre ondes modul\'ees 3) une \'etat
d'accrochage de fr\'equence \`a quatre ondes modul\'ees 4) l'\'etat
turbulent.
Ce mod\`ele, construit de mani\`ere heuristique, est ensuite justifi\'e
par des calculs analytiques et par des simulations num\'eriques.
Enfin, en annexe, le mod\`ele est reli\'e aux \'equations fluides des
plasmas.


\vspace{5mm}
\begin{center}
{\large\bf Abstract}
\end{center}

The problem attacked in this work is to provide models for the route
to turbulence 
observed in an experimental system. The system considered is a
magnetically confined discharge plasma. An array of 64 Langmuir probes
provides a spatiotemporal diagnostic. As a control parameter of the
system is varied, one observes different spatiotemporal regimes going
from a stable plasma to a state of weak turbulence.
A method for the analysis of those data is provided. It consists in
identifying modulated waves in the signals and is based on the
biorthogonal decomposition.
The description behaviour of the system in terms of modulated waves
allows to 
detect bifurcations which have not been observed before.
A model for the observed speed doubling phenomena is proposed. It is
based on a pure spatial modulation.
A second model is proposed to describe the whole
route to turbulence. 
More precisely, we give a dynamical system which describes the evolution
of temporal amplitudes of 
the modulated waves. It can be seen as a normal form for the sequence
of bifurcations observed, i.e., 1) the appearance of one modulated waves,
2) the four modulated waves quasiperiodic state 3) the four modulated
waves mode--locking  state 4) the turbulent state.
This model built in an heuristic way in then justified by analytical
calculus and numerical simulations. A connection of this model with
the plasma fluid equations is also provided in appendix.
\end{document}


Instabilities and transition to the weak spatiotemporal turbulence in
a plasma experiment.

Instabilities and transition to the weak spatiotemporal turbulence in
a plasma experiment.

Instabilities and transition to the weak spatiotemporal turbulence in
a plasma experiment.

Instabilities and transition to the weak spatiotemporal turbulence in
a plasma experiment.







\end{document}

Instabilit\'es et transition vers la turbulence
faible spatio-temporelle dans une exp\'erience de plasma

\vspace{10mm}
