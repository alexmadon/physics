% Upper-case    A B C D E F G H I J K L M N O P Q R S T U V W X Y Z
% Lower-case    a b c d e f g h i j k l m n o p q r s t u v w x y z
% Digits        0 1 2 3 4 5 6 7 8 9
% Exclamation   !           Double quote "          Hash (number) #
% Dollar        $           Percent      %          Ampersand     &
% Acute accent  '           Left paren   (          Right paren   )
% Asterisk      *           Plus         +          Comma         ,
% Minus         -           Point        .          Solidus       /
% Colon         :           Semicolon    ;          Less than     <
% Equals        =           Greater than >          Question mark ?
% At            @           Left bracket [          Backslash     \
% Right bracket ]           Circumflex   ^          Underscore    _
% Grave accent  `           Left brace   {          Vertical bar  |
% Right brace   }           Tilde        ~

\documentstyle[epsfig,elsartalex,12pt]{article}

\def\listdesfig{
A01Hfour.eps
A06Hfour.eps
A02Hfour.eps
A07Hfour.eps
A03Hfour.eps
A08Hfour.eps
A04Hfour.eps
A09Hfour.eps
A05Hfour.eps
A10Hfour.eps
poidsforme.eps
lock.eps
}

\def\subfigureA#1{
\leavevmode
\hbox{#1}
}

%%%%%%%%%%%%%%%%%%%%%%%%%%%%%%%%%%%%%%%%%%%%%%%%%%%%%%%%%%%%%%%%%%%%
\begin{document}
\begin{titlepage}
\begin{center}
\vspace{2cm}
{\bf \large A model for the bifurcations in plasma drift-waves}
\vspace{0.3cm}
\setcounter{footnote}{0}
\renewcommand{\thefootnote}{\arabic{footnote}}
{\bf Alex Madon}\footnote{CNRS, Centre de Physique Th\'eorique, Luminy,
         Case 907, F-13288 Marseille, France, 
         madon@cpt.univ-mrs.fr}
and {\bf Thomas Klinger}
\footnote{Institut f\"ur Experimentalphysik, 
Christian-Albrechts-Universit\"at,
         Kiel, Olshausenstrasse 40-60, D-24098 Kiel, Germany}
%
\today
\vspace{3cm}
{\bf Abstract}
\end{center}
Spatiotemporal data from a plasma drift--wave experiment are analyzed
by the biorthogonal decomposition (BOD). A describing of the route
to turbulence is given in terms of  modulated monochromatic 
travelling waves.
A low dimensional dynamical system decribing some of the features 
of the route to the weak turbulence observed is presented.
\vspace{3 cm}
Key-words : turbulence , drift-waves, 
bifurcation,
PACS : 52.35.R, 52.35.K, 47.20.K.
\vspace{2 cm}
\noindent Number of figures : 3
\end{titlepage}




\section{Introduction}\label{Introduction}
%%%%%%%%%%%%%%%

The understanding of the route to chaos and turbulence
for spatially extended systems is still a widely open question.
In many  systems, this route involves waves
(see \cite{Craick} and references therein).
Many theoretical approaches are based on wave interaction schemes
\cite{Craick}, for instance
the three wave interaction \cite{kaup79,chow95,Hasegawa77}.
Once the notion of modes is carefully defined, some phenomena
like period doubling, intermittency, and mode-locking
\cite{Jensen83a,Bishop86a,Stavans85}
can be understood in the framework of low-dimensional systems
theory. 


%plasma
Plasma turbulence  \cite{tsytovich} is considered  a
generic example of  spatiotemporal turbulence.
In particular, the drift-wave turbulence
\cite{horton84,horton90} is of high interest
for the understanding of anomalous transport in magnetically
confined plasmas \cite{hendel68,horton84,wagner93}

In this paper, we give a model of the route to 
weakly developed turbulence
observed in a plasma experiment.
We will refer to a recent publication
\cite{Madon96}
where a spatiotemporal model for a speed doubling
phenomenon was developed on the basis of experimental observations.
In the present case we are facing
a more complicated situation where several successive bifurcations occur
as the control parameter is increased,  finally leading to weakly 
developed turbulence behaviour.


In Section \ref{Experimental}  we present the experimental setup.
In this paper ten data sets are used.
Each of them describe the state of the system for a specific
control parameter value.
%*********
These data sets are then analyzed (Section \ref{analysis}) by using
the method of the Modulated 
Monochromatic Traveling Waves introduced in \cite{Madon96}. In
particular a mode-locking \cite{Arnold89} phenomenon is observed.
A low dimensional dynamical model system 
is constructed in Section \ref{Model} for the description
of the observed route to
turbulence. Indications about the dimension 
and bifurcations arising
in this experiment are given.


\section{Experimental Investigations}\label{Experimental}
%%%%%%%%%%%%
The experimental setup has been presented previously (see Ref.
\cite{Madon96,latten95}). It consists of three structurally and
functionally distinct regions : a central section and two
multidipole-confined discharge source chambers. 
For the present investigations, only one chamber is active.
Each chamber is separated from the central column by a biased grid.
The voltage of the grid $U_{g_1}$ separating the active chamber and the
midsection is the control parameter of the experiment. The voltage of
the second grid is set to zero. We chose to vary only the one
experimental parameter 
in order to simplify the analysis. The bias of the
injection grid is an accessible experimental parameter
that leaves the other plasma parameter almost unchanged.
The values of the other parameters can be found in \cite{Madon96}.

The main diagnostic tool is an azimuthally arranged multi-channel
Langmuir probe array \cite{latten95}. It is located in the middle of
the column, in a radial plane. Each Langmuir probe provides
the temporal fluctuations of the temporal plasma density $n_e$.
The total probe array allows one to measure with high precision
the temporal evolution of the spatial 
structure of drift waves which propagate predominantly in the azimuthal 
direction of the plasma column.
The temporal resolution is given by the maximum sample 
rate of the acquisition system ($\Delta t=1\mu{\rm s}$), and the spatial
resolution is given by the azimuthal angle between each two probes
($\Delta x=2\pi/64$). The data is stored as an $N \times M$-matrix
$u_{i,j}=n_{e}(i\Delta x,j\Delta t)$ where $N=64$ (space) and $M=2048$ 
(time). 


\begin{table}[htb]
 \begin{center}
  \caption{Values of the voltage $U_{g_1}$.}
  \label{values}
  \begin{tabular}{l|c|c|c|c|c|c|c|c|c|c}
label of the voltage     &$\epsilon_1$ &$\epsilon_2$ &$\epsilon_3$ &$\epsilon_4$ &$\epsilon_5$ &$\epsilon_6$ &$\epsilon_7$ &$\epsilon_8$ &$\epsilon_9$ &$\epsilon_{10}$\\
\hline
values of $U_{g_1}$(V) &5.8&5.9&6.1&6.5&7.1&7.6&7.8&8.0&8.4&9.8
  \end{tabular}
\end{center}
\end{table}

In the present paper, we refer to ten spatiotemporal data sets
corresponding to ten values of the control parameter $U_{g_1}$ noted
$\epsilon_n$, $n\in (1,\dots,10)$ (See
Tab. \ref{values} for the values).
Data sets are noted  ${\cal A}_n$ ($n=1,\dots,10$).
For each value of the control parameter $\epsilon_n$, 
the dynamical state of the physical system is described by
the experimentally obtained function $u^{\epsilon_n}(x,t)$.

\section{Analysis of the experimental data sets}\label{analysis}
%%%%%%%%%%%%%
The experimental data sets have been analyzed by using a method set up
in \cite{Madon96}. This method uses MMTW (Modulated Monochromatic
Traveling Waves) and is based on the BiOrthogonal
Decomposition BOD \cite{LimaCom,LimaSym,LimaMod} of the functions
$u^{\epsilon_n}(x,t)$. 
The BOD decomposes a function $u(x,t)$ into temporal  and 
spatial orthogonal modes and $u(x,t)$ can be written as follows :
\begin{equation}
u(x,t)=\sum \alpha_n \psi_n(t)\phi_n(x),
\end{equation}
with $\alpha_1\geq\alpha_2\geq\dots\geq 0$, and the orthogonality
 relations $(\phi_n,\phi_m)=\delta_{n,m}$ and 
$(\psi_n,\psi_m)=\delta_{n,m}$. The $\phi_n$ are called topos,
 and the $\psi_n$ chronos.
In \cite{Madon96}, we provided a method to group by pairs the
eigensolutions $(\alpha_n,\phi_n,\psi_m)$, such that the projected
signal defined by 
\begin{equation}
u_{m,n}(x,t)=a_m \psi_m(t)\phi_m(x)+a_n \psi_n(t)\phi_n(x),
\end{equation}
can be considered as a modulation (in a sense defined in
\cite{Madon96}) of a monochromatic  wave of wave number $k$
and (temporal) frequency $\omega$.


The dynamics
associated to the MMTW can be then described by a temporal
complexification $a_k(t)$ defined by :
\begin{equation}
a_k(t)=\alpha_m\psi_m(t)+i\alpha_n\psi_n(t)
\end{equation}
where $k$ is the wave number of the monochromatic traveling wave
associated to the MMTW.
In a similar way, spatial complexifications are defined by 
\begin{equation}
b_k(x)=\alpha_m\phi_m(x)+i\alpha_n\phi_n(x),
\end{equation}
and describe the spatial structure of the MMTW.

In this paper, we don't develop the method we used to group by pair
the eigenvectors (see Ref.\cite{Madon96} for a practical example of the
using of the method). We present only the evolution of some of the
quantities characterizing the MMTW's found in the experimental data
sets. 

The figure Fig.\ref{fftchro} shows the modulus squared Fourier
transforms  of the  four complexifications $a_k(t)$ ($k\in(1,\dots,4)$
for the ten values of 
the control parameter. The MMTW with $k$ numbers $(1,\dots,4$ are always the most energetic,
whatever the value of $\epsilon$ is. The Fourier transforms of the
complexifications 
$a_1$, $a_2$ and $a_4$ have not been plotted for the data set ${\cal
A}_1$ , because the energy of
the corresponding MMTW's are close to zero (see Fig.\ref{poidsforme}).
In the 
appendix \ref{MMTW} 
we show how the degeneracy of the energies of two MMTW's can result in a
mixing of two frequencies (i.e a strong modulation of the $a_k$'s).
This phenomenon can be observed 
in the Fourier transform of ${\cal A}_4$, ${\cal A}_5$  where
degeneracies have been observed for the MMTW's of number $k=1$ and
$k=2$ and for the MMTW's of number $k=3$ and $k=4$.

\begin{figure}
\begin{tabular}[t]{c c}
\centerline{\subfigureA{\epsfig{file={A01Hfour.eps},width=6truecm,height=2.4truecm}}\subfigureA{\epsfig{file={A06Hfour.eps},width=6truecm,height=2.4truecm}}}\\
\centerline{\subfigureA{\epsfig{file={A02Hfour.eps},width=6truecm,height=2.4truecm}}\subfigureA{\epsfig{file={A07Hfour.eps},width=6truecm,height=2.4truecm}}}\\
\centerline{\subfigureA{\epsfig{file={A03Hfour.eps},width=6truecm,height=2.4truecm}}\subfigureA{\epsfig{file={A08Hfour.eps},width=6truecm,height=2.4truecm}}}\\
\centerline{\subfigureA{\epsfig{file={A04Hfour.eps},width=6truecm,height=2.4truecm}}\subfigureA{\epsfig{file={A09Hfour.eps},width=6truecm,height=2.4truecm}}}\\
\centerline{\subfigureA{\epsfig{file={A05Hfour.eps},width=6truecm,height=2.4truecm}}\subfigureA{\epsfig{file={A10Hfour.eps},width=6truecm,height=2.4truecm}}}
\end{tabular} 
\caption{Modulus squared Fourier transforms of the complexifications $a_k(t)$,
$k=1,\dots,4$ for the ten data sets $A_1$ to
$A_{10}$. In each subfigure, the $k^{th}$ row corresponds to a spatial
frequency $k$. In the first subfigure, the Fourier transforms of the
functions $a_1$, $a_2$ and $a_4$ are set to zero. Frequency unit : $10
^{4}$ Hertz.}
\label{fftchro}
\end{figure}

The figure Fig.\ref{poidsforme} shows the evolution of the energy of
the waves of spatial frequencies $k=1, \dots,4$.
\begin{figure}[htb]
 \centering
 \epsfig{file={poidsforme.eps},width=10truecm,angle=-90} 
 \caption{Sketch of the evolution of the energies of the waves deduced
from the analysis of the experimental data sets.}
 \label{poidsforme}
\end{figure}
The sketch shown in Fig. \ref{lock} elucidates 
the evolution of the temporal frequencies 
as $\epsilon$ increases.
\begin{figure}[htb]
 \centering
 \epsfig{file={lock.eps},width=10truecm,angle=-90}

%bbllx=2.5cm,bblly=19.5cm,bburx=18.5cm,bbury=27.5cm}    
 \caption{Evolution of the temporal frequency with $\epsilon$.
Solid lines : mode m=1 and mode m=3. Dashed line : mode m=2.
Dotted line : mode m=4. Note the rapid increase of $\Omega_3$ just before
the bifurcation point.}
 \label{lock}
\end{figure}
%
 For the value
$\epsilon_1$ of
the control parameter, we practically have only one wave, which spatial
frequency is $k=3$ and temporal frequency is $\Omega_3$. For the value
$\epsilon_2$ , we have now four waves with spatial
frequencies $k=1$ to $k=4$. The frequency of the wave $k=3$ is
basically unchanged and equal to $\Omega_3$. The other
complexifications $a_k$ have temporal frequencies $\Omega_k$ with the
condition : $\Omega_2$ is $\Omega_3-\Omega_1$ and $\Omega_4$ is
$\Omega_3+\Omega_1$. When $\epsilon$ has the value $\epsilon_3$, the
state has the same properties. However, when $\epsilon=\epsilon_4$, a
mode locking phenomenon occurs. It corresponds to the the resonance
condition 
$\Omega_3-\Omega_1=2\Omega_1$.
The mode locked state disappears in the 
data set  ${\cal A}_6$. Indeed, in this data set,
a second frequency appears
associated with the mode of winding number $m=1$.
We $\epsilon$ is further increased, the modulations become stronger
and the systems goes to a drift-waves turbulence state.

\section{Model}\label{Model}
%%%%%%%%%%%%%
In this section, a set of model equations describing the observed 
 route to  weakly developed turbulence is presented.
The route to turbulence we described in the previous section can be
summarized as follows : 



\begin{center}
stable plasma\\
$\downarrow$\\
a ``mother'' wave with $k=3$, and frequency $\Omega_3$\\
$\downarrow$\\
two ``mother'' waves :\\
one with a $k=3$, and frequency $\Omega_3$\\
and a second with  $k=1$ and frequency$\Omega_1$\\
+two ``daughter'' waves :\\
one with $k=2$ and frequency $\Omega_3-\Omega_1$\\
and a second with $k=4$ and frequency $\Omega_3+\Omega_1$\\
$\downarrow$\\
Mode-locking\\
$\downarrow$\\
turbulence\\
\end{center}



The terms ``mother'' and ``daughter'' denote the fact that the sister
waves can not exist without the mother waves.
The model description of the physical phenomenon is 
mainly based on  the spatial and temporal complexifications.
Because of their predominant role in the overall dynamics, we
will only model the structures with a spatial winding number 1 to 4.
As  the experimentally obtained spatial structures are close to
perfect circles, we model the spatial complexifications
by simple exponentials $b_k(x)=e^{ikx}$ where $k$ is the
spatial winding number.
The temporal complexifications $a_k(t)$ are here modeled as the
solution of a dynamical system. This system reads

\begin{eqnarray}
\dot{a}_1&=&(i\omega_1+(\epsilon-\epsilon_{c_1}))a_1-r_{1,1}a_1\bar{a}_1a_1\nonumber\\
&&+ic^{(1)}_{\bar{2},3}\bar{a}_2{a}_3+ic^{(1)}_{\bar{1},2}\bar{a}_1a_2+ic^{(1)}_{\bar{3},4}\bar{a}_3a_4\label{eq1}\\
\dot{a}_2&=&ic^{(2)}_{\bar{1},3}\bar{a}_1{a}_3+ic^{(2)}_{1,1}a_1a_1+ic^{(2)}_{\bar{2},4}\bar{a}_2a_4\label{eq2}\\
\dot{a}_3&=&(i\omega_3+(\epsilon-\epsilon_{c_3}))a_3-r_{3,3}a_3\bar{a}_3a_3\nonumber\\
&&+ic^{(3)}_{1,2}a_1a_2+ic^{(3)}_{\bar{1},4}\bar{a}_1a_4\label{eq3}\\
\dot{a}_4&=&ic^{(4)}_{1,3}a_1a_3+ic^{(4)}_{2,2}a_2a_2\label{eq4}
\end{eqnarray}
where $\epsilon_{c_3}<\epsilon_{c_1}$ and $r_{i,i}$ are positive real 
numbers. $\epsilon_{c_3}$ and $\epsilon_{c_1}$ are the instability 
thresholds of the wave of winding number $3$ and $1$, and the
 $r_{i,i}$'s are related to the intensity of the nonlinear
saturation phenomenon.
Moreover we assume that the relation between $\omega_1$ and  $\omega_3$
is the same than the one given by experiment.
More precisely we assume that :

\beq\label{relatom}
\omega_3 \sim 2.5\omega_1
\eeq
The coupling coefficients $c^{(k)}_{i,j}$ are real.

\begin{rem}
This set of equations takes into account all possible
spatially resonant quadratic terms. 
The monomials of the type $u_{k'}u_{k''}$  are
called  spatially resonant with the
mode $k$ if $k'+k''=k$. 
The order of the resonance is 2. The cubic terms
are associated to resonances of order 3. The only cubic terms
written are the saturation terms of a Landau equation.
\end{rem}

\begin{rem}
With some additional hypotheses on the coupling coefficients,
this system can be split into one part which is hamiltonian
and another part which is not. All  monomials with a purely imaginary 
coefficient correspond then to the hamiltonian part, and the
monomials with a real coefficient (instability and saturation terms)
correspond to the non-hamiltonian part.
\end{rem}

In the following, we will look for the solutions
of (\ref{eq1}-\ref{eq4}) of the form

\beq\label{cyc}
a_k(t)=A_k e^{i\Omega_kt}
\eeq
where $A_k$ and $\Omega_k$ are two real numbers.

Let us study the case were $\epsilon<\epsilon_{c_3}$.
If we insert (\ref{cyc}) into the set of equations
(\ref{eq1}-\ref{eq2}) we get the condition
$A_1=A_2=A_3=A_4=0$. This solution is unique.
This state corresponds to the stable plasma state.
For $\epsilon_{c_3}<\epsilon<\epsilon_{c_1}$, we still have  the
previous solution but in the case where the coupling vanishes,
it is linearly unstable. We assume that it is also unstable
when the coupling is nonzero.

\begin{rem}
In this paper, we don't give the proof of the 
stability of the solutions we find. First numerical results on the
stability analysis of the mode-locked solution
indicate that Floquet exponents are very close to 1.
\end{rem}

Another state of type \ref{cyc} is defined by 
$A_1=A_2=A_4=0$ and $A_3=\sqrt{\epsilon-\epsilon_{c_3}}$ and
$\Omega_3=\omega_3$. In the case where the coupling is zero, 
it is well known (Landau equation, see \cite{hohenberg93} for instance)
that this state is stable.
We assume that it remains stable when the coupling is nonzero.
This state corresponds to the propagation of a single wave
with a spatial winding number three.

For $\epsilon<\epsilon_{c_1}$, we distinguish different cases.
Let us assume that the coefficients $c^{(i)}_{j,k}$ are of the same order.
As shown before, in the general case, the $A_k$'s and $\Omega_k$'s
are functions of $\epsilon$.
It is further assumed that there exists a range of
$\epsilon$ such that 

\beq\label{order}
a_3\gg a_1 \gg a_2 \sim a_4
\eeq
In this case we can simplify each equation of the system 
(\ref{eq1}-\ref{eq4})
and we obtain finally



\begin{eqnarray}
\dot{a}_1&=&(i\omega_1+(\epsilon-\epsilon_{c_1}))a_1-r_{1,1}a_1\bar{a}_1a_1\nonumber\\
&&+ic^{(1)}_{\bar{2},3}\bar{a}_2{a}_3+ic^{(1)}_{\bar{3},4}\bar{a}_3a_4\label{si1}\\
\dot{a}_2&=&ic^{(2)}_{\bar{1},3}\bar{a}_1{a}_3\label{si2}\\
\dot{a}_3&=&(i\omega_3+(\epsilon-\epsilon_{c_3}))a_3-r_{3,1}a_3\bar{a}_3a_3\nonumber\\
&&+ic^{(3)}_{1,2}a_1a_2+ic^{(3)}_{\bar{1},4}\bar{a}_1a_4\label{si3}\\
\dot{a}_4&=&ic^{(4)}_{1,3}a_1a_3\label{si4}
\end{eqnarray}

Inserting the trial solution (\ref{cyc}) into 
the system of equations (\ref{si1}--\ref{si4}) we obtain the following
information about the cycle states :

\begin{enumerate}
\item {\bf Locking conditions.} Because it was assumed
that the $\Omega_k$ are numbers independent on $t$,
the equations (\ref{si1}--\ref{si4}) imply a sequence of relations between
the frequencies that reduce to the following two conditions :

\beq\label{reom1}
\Omega_2=\Omega_3-\Omega_1
\eeq

and 

\beq\label{reom2}
\Omega_4=\Omega_3+\Omega_1
\eeq

\begin{rem}
When the relation order (\ref{order}) 
is no longer valid, we get more relations
which imply a stronger condition on locking.
\end{rem}

\item {\bf The amplitudes $A_1$ and $A_3$}. 
Using the form of the trial solution
(\ref{cyc}) and projecting (\ref{si1}) and (\ref{si3}) onto the real
axis, we get 

\beq\label{a1}
A_1=\sqrt{\frac{\epsilon-\epsilon_{c_1}}{r_{1,1}}}
\eeq
and
\beq\label{a2}
A_3=\sqrt{\frac{\epsilon-\epsilon_{c_3}}{r_{3,3}}}.
\eeq

\begin{rem}
This result gives a posteriori the range of $\epsilon$ where the
oder relation  $a_3\gg a_1$ is valid. It corresponds to  values of
$\epsilon$ slightly larger than $\epsilon_{c_1}$. 
\end{rem}

\begin{rem}
Note that since the equations for $a_2$ and $a_4$ have no
dissipation, we can not evaluate the amplitudes $A_2$ and $A_4$
in the same way we have evaluated the amplitudes $A_1$ and $A_3$.
\end{rem}


\item {\bf Supplementary relations.} Using  (\ref{cyc})
and projecting the system of equations (\ref{si1}-\ref{si4}) onto the 
imaginary axis we get


\begin{eqnarray}
\Omega_1{A}_1&=&\omega_1A_1+c^{(1)}_{\bar{2},3}\bar{A}_2{A}_3+c^{(1)}_{\bar{3},4}\bar{A}_3A_4\\
\Omega_2{A}_2&=&c^{(2)}_{\bar{1},3}\bar{A}_1{A}_3\\
\Omega_3{A}_3&=&\omega_3A_3+c^{(3)}_{1,2}A_1A_2+c^{(3)}_{\bar{1},4}\bar{A}_1A_4\\
\Omega_4{A}_4&=&c^{(4)}_{1,3}A_1A_3
\end{eqnarray}

This set of equations requires numerical treatment. In general
the frequencies $\Omega_1$ and  $\Omega_3$ observed for an 
$\epsilon$ larger than $\epsilon_{c_1}$ can be significantly
different  from the linear values  $\omega_1$ and  $\omega_3$.

\begin{rem}
It cannot be excluded  that the  heuristic relation 
for the frequencies \ref{relatom} is not the optimum for the model.
\end{rem}
\end{enumerate}
As $\epsilon$ increases the relation
$a_3 \gg a_1$ is violated.
In that case, we need to take into account the complete system 
of equations (\ref{eq1}-\ref{eq4}).
The condition that the $\Omega_k$'s are independent of $t$
implies then the stronger relation

\beq\label{relom}
\Omega_k=k\Omega_1
\eeq


\begin{rem}
The conditions (\ref{reom1}) and (\ref{reom2}) are 
 a special case of the relation (\ref{relom}).
\end{rem}
Projections onto the complex and real axes result in relations
between the $A_k$'s and $\Omega_k$'s.
Though we do not give these relations here, it is important
to note that the state is now a mode-locked state, that is,
a state where the frequencies obey relation (\ref{relom}).

\begin{rem}
Depending on the values of the $c^{(i)}_{j,k}$, the value of $A_2$
can perform a sudden jump to mode-locking as observed
in the experiment.
\end{rem}




\section{Conclusion}


We have analyzed the route to weakly developed
turbulence for a drift-wave plasma experiment.
A set of model equations  describing this route
is proposed.
The model is not derived from a basic set of the plasma equations.
It is based on the notion of spatial resonance
and on general ideas of bifurcation theory.
A set of ordinary differential equation s
describes the observed relation
between the temporal frequencies of the four first monochromatic waves
observed in experiment.
It describes furthermore the successive occurrence of different waves.

{\bf acknowledgment}
One of us (A.M)  acknowledges Prof. R. Lima for  helpful discussions.
A.M. thank also Prof. A. Piel of the Institut
f\"ur Experimentalphysik at Kiel and all his team for the 
warm hospitality during part of this work.
Dr. A. Latten is specially thanked for his patience when
during the  recording of data sets. We thank also Prof. L. Lellouch
for the corrections done on the manuscript.


\appendix

\section{MMTW and Degeneracy}\label{MMTW}
%%%%%%%%%
In Section \ref{analysis} it was stated
that strong modulations occur
when the eigenvalues are degenerate.
In this appendix, it will be shown how the degeneracy
of the energy of MMTW can induce such
modulations.

It is well known in the theory of 
the spectral analysis of linear operators
that if some eigenvalues are degenerate, each vector $\phi$
of the corresponding eigenspace $E$ is an eigenvector. 
If $R$ is an orthogonal operator which leaves $E$ globally
invariant and if $\phi$ is an eigenvector it follows that $R\phi$
is also an eigenvector.
Using the correlation operator $U^+U$ it is
sufficient to understand why in the case of two waves with
the same energy  one observes
strong modulations, i.e.,  rotations in the
functional space $\chi(X)$. We give here a theorem
associating the modulations in the space $\chi(T)$ with
those in the space $\chi(X)$.

Let us suppose that there exist $N$ eigensets $(a_i,\phi_i,\psi_i)$,
 $1\leq i \leq N$
with the same eigenvalues $a_i=a$ for  $1\leq i \leq N$,


\beq
U\phi_i=a\psi_i
\eeq
 
\beq
U^+\psi_i=a\phi_i
\eeq
 
for  $i\in\{1,\dots, N\}$
 
Let $R$ be the operator acting in $H(T)$, whose restriction to the
eigenspace $E_T$ spanned by the vectors $\psi_i$, is orthogonal
and leaves $E_X$ invariant
 
\beq
R^+R=1
\eeq

\begin{thm}
If $(a,\phi_i,\psi_i)$ is an eigenset of one
eigenvalue and corresponding eigenvectors, then there exists
an operator $Q$ such that $(a,Q\psi_i,R\phi_i)$ is also an eigenset.
$Q$ is defined by
 
\beq
<\phi_m|Q|\phi_n>=R_{m n}
\eeq
where $R_{mn}=<\psi_m|R|\psi_n>$.
\end{thm}


\begin{pf}
Let us evaluate $U^+R\psi_i$ :
 
\beq
U^+R|\psi_i>=\sum_{m,n}U^+|\psi_m>R_{m n}<\psi_n|\psi_i>
\eeq
 
 
\beq
U^+R|\psi_i>=\sum_{m,n}U^+|\psi_m>R_{m n}\delta_{n,i}
\eeq
 
 
\beq
U^+R|\psi_i>=\sum_{m}U^+|\psi_m>R_{m i}
\eeq
 
Using the definition of the eigenvectors

\beq
U^+R|\psi_i>=a\sum_{m}R_{m i}|\phi_m>
\label{ex2}
\eeq

we define the operator $Q$ by


\beq
<\phi_m|Q|\phi_n>=R_{m n}
\eeq
 

or equivalently by 

\beq
Q=\sum_{m,n}|\phi_m>R_{m n}<\phi_n|.
\eeq
 
 

Then let us evaluate the action of $Q$ on the eigenvector $\phi_i$

 
\beq
Q|\phi_i>=\sum_{m,n}|\phi_m>R_{m n}<\phi_n|\phi_i>
\eeq
 
\beq
Q|\phi_i>=\sum_{m,n}|\phi_m>R_{m n}\delta_{n,i}
\eeq
 
\beq
Q|\phi_i>=\sum_{m}R_{m i}|\phi_m>.
\label{ex1}
\eeq
 
By identification of \ref{ex1} and \ref{ex2} it follows that 
 
\beq
U^+R|\psi_i>=a Q|\phi_i>.
\eeq
 
We show in the same way that
 
\beq
U Q|\phi_i>=a R|\psi_i>.
\eeq
 
\end{pf}



\begin{rem}
Let  $E_X$ the space spanned by the $|\phi_i>$, $1\leq i \leq N$ and
$E_T$ the space spanned by the $|\psi_i>$, $1\leq i \leq N$.
The restriction $U_r$ of $U$ to the eigenspace $E_X$, $E_T$
has a symmetry in the sense of Ref. \cite{LimaSym}, 
that is, there exists
two operators $Q$ and $R$ such that :

 
\beq
U_r Q=R U_r
\label{sym}
\eeq
 
This can be seen by writing the operator $U_r$ in 
Dirac's notation a follows

\beq
U_r=\sum_i|\psi_i>a<\phi_i|
\eeq
 
and
 
\beq
U_r=\sum_i R|\psi_i>a<\phi_i|Q^+.
\eeq
 
Thus
 
\beq
U_r=R U_r Q^+
\eeq
 
which is equivalent to (\ref{sym}).

\end{rem}
%%%%%%%%%%%%%%%%%%%%%%%%%%%%%%%%%%%%%%
In the case where two waves have the same energy the
spectral problem has a degeneracy equal to four.
So instead of observing  only one
frequency in the eigenvectors, we observe two.


\begin{rem}
The rotation allowed when the eigenvalues
are degenerate can be used to model certain bifurcation s
\cite{LimaSym,LimaMod}.
\end{rem}




\bibliographystyle{unsrt}
\bibliography{papMadKli2}

\end{document}






\end{document}